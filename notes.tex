\documentclass[twoside]{article}
\usepackage[b4paper, margin=14.82mm, includehead]{geometry}
\usepackage{changepage}
\usepackage{enumitem}
\usepackage{fancyhdr}
\usepackage{fontspec}
\usepackage{graphicx}
\usepackage{hanging}
\usepackage{multicol}
\usepackage{musixtex}
\usepackage{wasysym}
\setmainfont[
 BoldFont={[ACADEMICO-BOLD.otf]}, 
 ItalicFont={[ACADEMICO-ITALIC.otf]},
 BoldItalicFont={[ACADEMICO-BOLDITALIC.otf]}
 ]{[ACADEMICO-REGULAR.otf]}
\setlength{\columnsep}{1cm}

\pagestyle{fancy}
\renewcommand{\headrulewidth}{0pt}
\fancyhf{}%clear all headers and footers
\fancyhead[LE]{\fontsize{8pt}{10pt}\selectfont \slshape\rightmark}
\fancyhead[RO]{\fontsize{8pt}{10pt}\selectfont \slshape\leftmark}
\fancyhead[LE,RO]{\thepage}
\setcounter{page}{90}

\newcommand\dynmark[1]{\scalebox{0.9}{#1}{\kern1pt}}

\begin{document}
\begin{center}
\underline{\huge{Editorial Commentary}}
\end{center}

This edition is based on 4 sources: the 1924 hand-copied orchestral score published by Heugel, the 1925 2-piano reduction published by Heugel, the 1924 manuscript of an earlier version for piano and 12 instruments, and the 1924 reduction of \textit{Adagio} for violin and piano published by Heugel. There are numerous differences between the sources, and the timeline of revisions between the sources is unclear. This edition is primarily based on the orchestral score, excepting the piano part, which is primarily based on the 2-piano reduction. The manuscript and \textit{Adagio} are generally only used to resolve potential errors. Listed below are notable errors and discrepancies. Those of greater significance are bolded and italicized.

\begin{hangparas}{15pt}{1}
\bigbreak
OS = Orchestral Score. 2P = 2-Piano Reduction. MS = Manuscript. AD = \textit{Adagio}.
\begin{multicols}{2}

\underline{General}

Trills are broken at barlines and restated without ties in the original sources. These are changed to continuous trills with ties.

I.9: The lower part of the accompaniment in 2P does not correspond to anything in OS or MS.

I.15: Reducing dynamic from \dynmark{\p} to \dynmark{\pp} to make room for \dynmark{p} at 16 and to match 2P and 115

I.36--37: 2P accompaniment upper part corresponds to MS Violin II part at 36. 37 corresponds to nothing in OS or MS. Lower part corresponds to MS Cellos and Basses.

I.39: ``Cédez un peu'' only in OS

I.93: An \dynmark{\f} from each of the dynamics is crossed out in OS, making it 1 level lower than 2P. The cross outs are ignored.

I.100: Adding \dynmark{\fff} found only in 2P

I.101: Moving all dynamics to beat 1.5 to be more climactic

II: Second hand $\quarternote=42$ in OS

II.39: ``Rall.'' is at beat 1 in OS, at beat 2 in 2P.

III.17: The melody in the second half of this measure in 2P is only in the MS Violin I part. Removal from OS seems intentional.

III.56: ``poco riten.'' in OS. ``rit.'' in 2P

III.57: Using \dynmark{(\p)} in place of \dynmark{\p} that has been placed on some parts seemingly arbitrarily

III.57: ``a Tempo'' only in 2P

III.104: ``poco riten.'' in OS. ``rit.'' in 2P

III.111: Respelling D♯ major to E♭ major at beat 3

III.179: The accompaniment melody is marked with ``Htb.'' in 2P when the melody is played by Violins II in OS (Saxophone/Oboe in MS).

III.214: The final note is a held fermata quarter note in OS that seems to have been changed from an eighth note. It is an eighth note without fermata in 2P. It is a fermata dotted quarter note in MS.

\underline{Flutes}

I.3, 103: Trill extends over sixteenth note on beat 3 in OS. Trill does not extend to this note in MS, and at 49.

I.10: Flute 1 has no explicit \dynmark{\mf}. Assuming it from Flute 2

I.23--34: 2P largely corresponds to MS here, not OS. See Oboe and Bassoon also.

I.53: Hairpin is before this group of notes in OS, under the group of notes in MS.

I.113--114: Making beaming consistent with 13 and MS.

III.28, 200: Only the run at 200 has E♭ marked. It seems likely that the C(♯)s and F(♯)s should be marked natural.

III.91: Slur extends to end of system, but there is no leading slur in next system.

III.206: Missing ♮ on C for Flute 2 in OS

\underline{Oboe}

Corresponds to part in MS labeled for Clarinet in Bb, with Saxophone label crossed out.

I.23--34: 2P largely corresponds to the manuscript here, not OS. See Flute and Bassoon also.

I.81: Moving \dynmark{\mf} from start of measure to beat 2 where the G starts

III.12: \dynmark{\f} has been added in second hand in OS. Should likely be \dynmark{\ff} like other parts. \dynmark{\ff} in MS

III.30: F likely missing on beat 6. It is present in MS and at 202.

\textbf{\textit{III.105: OS has E♭ instead of E whereas 2P and MS have E explicitly marked with ♮. ♭ in OS appears to have been added by a second hand. See Piano and Violins I also.}}

\columnbreak

\underline{Bassoons}

\textbf{\textit{MS has 1 bassoon. In OS, the ``2'' in ``2 Bassons'' seems to have been added at a later point. Indications of ``à 2'' and ``1°'' are inconsistent. For these reasons, only a single part for Bassons 1.2 has been made.}} The second bassoon was perhaps added to support historically weaker bassoons.

I.23--34: 2P largely corresponds to the manuscript here, not OS. Bassoon solo only in MS and 2P. See Flute and Oboe also.

I.71-73: 2P accompaniment lower part corresponds to a bassoon solo that is only in MS.

I.119: Changing \dynmark{\f} to \dynmark{\ff} to balance with low strings

III.2: Removing staccato on first note that no other part has and which is not elsewhere in the bassoon part

III.18: First note E in OS should be F♯.

\underline{Horns}

I.11: Last note that is B in OS should be C♯ based on MS and measure 111 in OS.

II.3--4: Trailing slur at end of measure 3, but slur does not lead from previous system at measure 4 in OS. Slur only over measure 4 in MS

III.1--3, 173--175: 2P accompaniment top line specifies ``Cors'' and corresponds to horn parts in MS only.

III.31--32: Changing articulation to match other parts. No accent at 31 beat 1, and staccato instead of accent at 32 beat 4 originally

III.37: Some accents on beat 4 seem to be missing.

\underline{Trumpet}

MS has part for Trumpet in C

I.58: Last sixteenth has ♮ on E instead of ♭ in OS

I.107: Slur between A and D unlikely to be intended. Similar section at 6 does not have a slur.

I.120: Adding accent like Violins I

III.60: Beat 3 Concert G in OS and 2P. G♭ in MS

III.75--79: This part has been crossed out in OS. It is still present in MS and 2P.

\textbf{\textit{III.106: Beat 6 Concert G♭ in OS, ♭ seemingly drawn by a second hand. G in 2P and MS. G fits context better.}}

\textbf{\textit{III.107: Concert G in OS and MS, G♭ in 2P. G fits context better.}}

\textbf{\textit{III.108: Concert A♭ in OS, ♭ seemingly drawn by a second hand. A in 2P and MS. A fits context better.}}

III.131: Concert F♯ in OS should be E.

III.131: See Timpani.

III.171: Adding accent

III.214: Written a step too high in OS

\underline{Timpani}

I.9: Adding accent, as in MS

III.131: The hairpin here is aligned with the Trumpet dynamic but it seems more logically applied to Timpani.

\underline{Piano}

I.3: Only instance where trilled note is tied to the following sixteenth in 2P. Deleting tie

I.6--13: Dynamics only in OS

I.10: Higher octave is missing present in OS

I.11--13: Dynamics only in OS

I.13: ♯ on G♯ missing in 2P

I.15: 2P marked at \dynmark{\pp}. OS marked at \dynmark{\p} level with the strings

I.15-17: 2P doesn't have ``cresc.'' until 17 beat 3.5

I.18: Last Gs have ♯s in OS. Based on similar section at 118 and 2P and OS, they should be G♮s.

I.19: Ossia as in OS. MS has everything that is in both OS and 2P.

I.21: Beat 1.5 is G3 in OS. A3 in 2P and MS. Similar section at 121 has A3, so this is likely an error.

I.21: Beat 2.5 missing A4 in OS. It is present at 121.

I.25: Beat 1 lowest note of LH chord is A in OS. Likely an error as it is B in 2P and MS

I.25: Beat 2 of RH is E in OS. Likely an error as it is D in 2P and MS

I.27: Hairpin only in 2P

I.31: Hairpin only in 2P

I.32: Beat 2.5 of RH is D in OS. Likely an error as it is C♯ in 2P and MS

I.33: \dynmark{\pp} only in OS and MS. Hairpin only in OS

I.34: \dynmark{\f} at beat 1 in OS, at 1.5 in 2P. Other instruments \dynmark{\f} at beat 1.5

I.38: Beat 2 missing ♭ on E♭ in OS. It is present in 2P and MS.

I.39--40: Decrescendo only in OS. \dynmark{\pp} only in 2P

I.42: Beat 2.5 lower note of LH is F♮ in OS. Both 2P and MS have G. Due to similar nearby chords with lower note G in OS, this is likely an error despite the explicit ♮.

I.47: \dynmark{\p} only in 2P. This matches the dynamics of orchestra in OS.

I.55--57: Dynamics only in 2P

I.65: Beat 4.5 A4 of LH missing in OS. It appears to have been added as a correction in MS.

I.67: Beat 1 ossia is as in OS. This is how it is originally in MS, but has been crossed out for how it is in 2P.

I.71: \dynmark{\p} in OS. \dynmark{\pp} in OS

I.71: Beat 3.75 G missing in OS. Likely an error as it is in 2P and MS

I.72: Beats 2.75 and 3.75 OS missing ♮s on F♮s.

I.79: \dynmark{\pp} only in 2P

I.84+: Beats 2 and 4, and beat 1 of 85 LH has F♯4s in 2P and MS; and G4s in OS. Perhaps these are over-corrections in OS to be an octave below RH.

I.85: \dynmark{\pp} in OS. \dynmark{\p} in 2P

I.89: D is marked as D♯ in 2P, but D with explicit ♮ in OS and MS, so this is likely an error. Flutes play D(♮) nearby.

I.92: Hairpin only in OS

I.92: Beat 4.5 B missing ♮ in OS

I.94: Beat 2.75 A instead of F in LH 2P. This is likely an error based on nearby patterns.

I.95: Last E♭ missing ♭ in OS

I.97: Beat 3 missing in OS

I.113: \dynmark{\f} only in 2P at beat 1.5. Moving to beat 1 like at measure 13

I.119: \dynmark{\ff} only in 2P. This is level with orchestra in OS.

I.119: Only MS has the D5 at beat 4.5 like at 19

I.121: Using the same voice-note distribution as 21

I.121: Beat 3.5 OS missing E. It is present at 21

II.8: \dynmark{\mf} only in 2P marked below \dynmark{\f} accompaniment. Orchestra melody plays \dynmark{\mezzopiano} in OS.

II.12: Beat 3 OS missing E

II.12--13: Crescendo is at 12 in 2P for piano and accompaniment. Crescendo is at measure 13 in OS. Using only OS crescendo

II.18: \dynmark{\f} in 2P below the accompaniment at \dynmark{\ff}. \dynmark{\ff} in OS above the accompaniment at \dynmark{\f}.

II.23: Lower note of OS is E2. Likely supposed to be C♯3.

II.25--26: Decrescendo over the 2 measures in 2P only over last beat of 26 in OS. Changing to be even with orchestra decrescendos in OS

II.27: \dynmark{\pp} in 2P with \dynmark{\mezzopiano} accompaniment. \dynmark{\p} in OS, orchestra at \dynmark{\p}. \dynmark{\pp} seems more appropriate to allow for Oboe decrescendo from \dynmark{\p} into 31

II.27: Missing ♮ on F♮ in RH in 2P and MS

\textbf{\textit{II.28: The FABC chord in 2P is seemingly an error. Compare to adjacent chord, and the chord as in OS, MS, and AD. However, it seems to be the extant way this is played, and the other sources are more aligned with each other than with 2P in this section.}}

II.29: Beat 3.5 seems to have been written in as a correction to match the 2P, MS, and AD. However, only the D has been written in.

II.33--34: Dynamics only in 2P. They are placed below both voices but it seems illogical for it to apply to the LH pattern, since it is in unison with Flute 1 that has no crescendo.

III.1, 4--5, 173, 180: The quarter notes in 2P and MS are written as eighths in OS.

III.7: Second half of measure for LH is missing in 2P. See 179

III.8: \dynmark{\mf} only in OS

III.11: \dynmark{\ff} only in 2P

III.15: \dynmark{\f} only in 2P	

III.17: \dynmark{\p} only in 2P

III.17: ♯ on G♯ LH missing in OS

III.19--21: Dynamics only in OS

III.23: ``cresc.'' only in 2P

III.25: \dynmark{\f} in 2P. \dynmark{\ff} is OS

III.29: \dynmark{\ff} only in OS

III.34: ♮ missing on C♮ in 2P

III.38--39: OS has inconsistencies in the pattern, likely mistakes. D4+C♯5 instead of D4 at 38 beat 1.5, 39 beats 1.5 and 3.5. C♯5 instead of D4 at 38.3.5

III.44: Beat 4 quarter note in 2P, eighth note in OS and MS

III.48: Beat 1 E4 missing in OS.

III.49: No dynamic in OS. 2P has \dynmark{\pp} at same level as accompaniment. Using \dynmark{\p} to match orchestra

III.50--51: Dynamics only in OS

III.51: Beat 3 ♮ on C♮ missing in 2P

III.52: Beat 3 OS has C5. Likely an error as it is A4 in 2P and MS

III.52: Beat 6 OS has LH duplicated into RH 8\textsuperscript{va} staff, likely an error. Missing ♮ on F on LH staff

III.53--54: Missing middle octaves probably errors

III.53--54: From 53 beat 5, OS is written an octave too high. (2P has 8\textsuperscript{va} bassa on a treble clef to show the correct octave)

III.57: No dynamic in OS. 2P has \dynmark{\pp} at same level as accompaniment. Using \dynmark{\p} to match orchestra

III.61: OS has E♭ whereas 2P and MS have E♮ explicitly marked. E♮ matches horns and Violins II, so the ♭ is likely an error.

III.61--79: Dynamics only in 2P. Balancing them with OS orchestra dynamics

III.68: Beat 4 LH chord should likely have B♭ instead of B, as in previous measures, but neither 2P nor OS has a ♭ on this note. (B is not present in MS.)

III.71: The trill starts at beat 4 in OS, at beat 2 in 2P

III.74: ♮ on F♮ missing in OS

III.74: A4 instead of G4 in OS and MS. Based on general motivic shapes, it is likely an error in 2P.

III.79: Beat 6 additional G4 in OS and MS. MS additionally has a D4 here. Based on similar measure at 17, these should both be included.

III.80: Beat 1 additional E3 in OS and MS. Based on similar measure 18, it is likely an error in 2P.

III.80: Beat 6 A4 instead of G4 in OS and MS. Based on similar measure 18, it is likely an error in 2P.

III.83: Beat 6 ♮ missing on F♮ in OS

\textbf{\textit{III.86: Beat 6.5 should likely be C♮ instead of C♯ but no source has a ♮ on this note.}}

III.91: Beat 5 has just a quarter note in OS LH. Likely an error based on similar measures

III.92: Beat 5 E missing in OS RH. Likely an error based on similar measures

III.97: G(♮)s in OS, G♯s in 2P. MS has G(♮)s but also D(♮)s instead of D♯s. Based on context, the ♯ is likely missing in OS.

III.98: Beat 3.5 F♮ in OS, but no accidental in 2P or MS, so this is likely an error.

III.99: The RH triad is C♮+E+G♮ in 2P, C♮+E♯+G in OS, C♮+E+G♯ in MS. Based on context, MS is correct.

III.101: Dynamic only in 2P

III.101--102: Compared to 2P, MS is missing octave starting with G4 from 101 beat 5-102 beat 1. OS is missing the octave starting with G5 from 101 beat 5 to 102 beat 6.

III.105: Dynamic only in 2P

\textbf{\textit{III.105: OS has E♭ whereas 2P and MS E♮ explicitly marked. ♭ in OS appears to have been added afterwards. See Oboe and Violins I.}}

III.111: All parallel major chords in OS and MS. No ♯ on E on beat 1, F marked ♮ on beat 2, and no ♯ on A on beat 3 breaks this pattern in 2P, so these are likely errors.

III.114: F and middle voice missing in second half of measure in OS, which are in 2P and MS.

\textbf{\textit{III.119: The trill is on the C♮ in MS, instead of on the G as in 2P and OS, forming a continuous trill with the LH C♮ in the following measures.}}

III.123: Lowest note in RH in OS should likely be B instead of A based on 2P and MS.

III.132: RH staff in OS missing 8\textsuperscript{va} line.

III.134: Based on missing ♮ on C at beat 3 in 2P, a C♮ at beat 1 was likely intended.

III.161: Beat 2 has ♭ is written on D instead of B♭ in OS.

III.163: Beat 2 OS has F♯5 instead of G5.

III.166--167: The beam splits here in 2P just seem to be because MS split the beam to avoid colliding with an 8\textsuperscript{va} line.

III.168--169: Measures are empty (no rests) in OS. Likely the chords were mistakenly left out.

III.173: \dynmark{\mezzopiano} in 2P, \dynmark{\mf} at beat 3 in OS. Accent in OS and MS missing in 2P

III.181: Beat 6 OS has F♯ instead of G

III.189: No E4 in OS

III.189--195: Dynamics only in 2P

III.191: D6 in OS. Likely an error as 2P and MS have C♯. Also C♯ in OS at measure 19

III.197: \dynmark{\f} in 2P. \dynmark{\ff} in OS

III.199: E♮4s in 2P. E♭4s in OS. MS is illegible. Violas have E♮4. Similar measure at 27 has E♮ in all sources.

\textbf{\textit{III.203: The second half of the measure is a third (diatonic B♭ major) higher in OS, which matches MS. However, similar measure 30 in OS matches 2P, with MS a third higher. It seems likely this was a change only partially carried through.}}

III.207: Beat 3 D4 missing in OS

\underline{Violins I}

I.2: Adding staccatos

III.27: Beat 5 adding accent

III.41: Beat 5 in OS has D and F♯ (i.e. one was corrected to the other but the order is unclear). Likely should be F♯ based on similar sections

\textbf{\textit{III.105: OS has E♭ instead of E whereas 2P and MS have E explicitly marked natural. ♭ in OS appears to have been added afterwards. See Oboe and Piano.}}

III.99: Adding staccato like Violins II and Violas

III.106: Missing ♭ on A♭ based on 2P and context

\columnbreak

\underline{Violins II}

I.40: The notes are in different voices, but ``div.'' is not explicitly written.

I.56--57: Normalizing articulation with Violas

III.27: Beat 5 adding accent

III.41: Beat 5 in OS has D and F♯ (i.e. one was corrected to the other but the order is unclear). Likely should be F♯ based on similar sections

III.106: Beat 4 2P has G♭ here instead of G, but horns have D (concert G).

III.203: Last 2 notes have accents instead of staccatos.

\underline{Violas}

\textbf{\textit{I.57: Gs in OS where 2P and MS have G♯.}}

III.8: Second chord has G+D, likely supposed to be A+D.

III.75: Adding staccatos to last 2 notes like Flute 1 and Violins II at 77.

III.76: A slur over the last 2 notes seems to have been erased and replaced with staccatos. This is inconsistent with Flute 1 and Violins II at 78.

III.175: Beat 3 in OS has C♯. Based on measure 3 and MS, this should be D.

\textbf{\textit{III.179: Marked ``div.'' but this seems unlikely. The similar section at 7 is not marked ``div.'' and has brackets suggesting non div. In addition, after the system break at 181, there is a bracket suggesting non div.}}

III.196: Trailing slur, but no leading slur on next system. Next set of notes is slurred, so slur is unlikely.

III.202: Beat 6 in OS maybe should be F♮ instead of E♭, as at measure 30 and in MS.

III.214: Half note should be dotted quarter note.

\underline{Cellos}

I.100: Adding accent on last note

II.35: There is a slur over these notes in the MS and 2P but not in OS.

\textbf{\textit{III.76: The chord at beat 5 should potentially be on beat 4 like the surrounding measures, but the writing in OS looks deliberate. This section is not included in 2P and the section in MS is too dissimilar to be any indication.}}

III.211--213: Adding accents as in previous measures

\underline{Basses}

I.100: Adding accent on last note

II.35: There is a slur over these notes in MS and 2P but not in OS.

III.44: Adding accent based on surroundings

III.49: ``col legno'' likely intended as with other strings and at 97

III.179--183: Adding staccatos

III.187: Adding accent based on surroundings

III.213: Adding accents as in previous measures

\end{multicols}

\end{hangparas}

\end{document}
